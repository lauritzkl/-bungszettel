\setcounter{punkte}{5}
\begin{aufgabe}{Teilchen in 3D}
Die Trajektorie eines Teilchens mit Masse m im dreidimensionalen Raum sei in sphärischen
Polarkoordinaten gegeben durch:
\begin{equation}
  \vec{r} =
  \begin{pmatrix}
      R(t) \sin\!\left(\vartheta(t)\right) \cos\!\left(\varphi(t)\right) \\
      R(t) \sin\!\left(\vartheta(t)\right) \sin\!\left(\varphi(t)\right) \\
      R(t) \cos\!\left(\vartheta(t)\right)
  \end{pmatrix}
  \label{eqn:trajektorie}
\end{equation}
\begin{description}
  \item[(a)] Bestimmen Sie die Geschwindigkeit $\dot{\vec{r}}$ und die Beschleunigung
  $\ddot{\vec{r}}$ des Teilchens.
  \item[(b)] Unter welcher Bedingung gilt $\vec{r} \perp \dot{\vec{r}}$ ?
  \item[(c)] Berechnen Sie die kinetische Energie des Teilchens:
  \begin{equation}
    E = \frac{1}{2} m {\dot{\vec{r}}\,}^2
    \label{eqn:ekin}
  \end{equation}
\end{description}
\end{aufgabe}

% \loesung{
%
% }
%
% \setcounter{punkte}{5}
