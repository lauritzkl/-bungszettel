\setcounter{punkte}{5}
\begin{aufgabe}{Schräger Wurf}

Betrachten Sie den zweidimensionalen schrägen Wurf. Zum Zeitpunkt $t_0 = \SI{0}{\second}$
befindet sich ein Teilchen in den Koordinaten $x_0 = \SI{5}{\meter}$ und $y_0 = h = \SI{5}{\meter}$.

\renewcommand{\labelenumi}{\alph{enumi})}
\renewcommand{\labelenumii}{(\roman{enumii})}
\begin{enumerate}
    \item Wie sieht die Bewegung für $v_x \neq 0$ und $v_y \neq 0$ aus? \\
          Beschreiben Sie die Bahnkurve $\vec{r}$ des Teilchens in Abhängigkeit von $x$.
          Skizzieren Sie die beschriebene Situation.
    \item Wo befindet sich das Teilchen für $x = \SI{10}{\meter}$, wenn $v_x = \SI{5}{\meter\per\second}$
          und $v_y = \SI{-7}{\meter\per\second}$ betragen?
    \item Diskutieren Sie die Fälle:
      \begin{enumerate}
        \item $v_x > 0$ und $v_y > 0$
        \item $v_x > 0$ und $v_y = 0$
        \item $v_x = v_y = 0$
      \end{enumerate}
          Welchen Bewegungen entsprechen die Fälle (i) - (iii)?
\end{enumerate}


\end{aufgabe}

\loesung{

}

\setcounter{punkte}{5}
