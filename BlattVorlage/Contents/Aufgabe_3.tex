\newpage
\setcounter{punkte}{4}
\begin{aufgabe}{Harmonischer Oszillator}
Ein punktförmiges Teilchen der Masse m bewege sich in dem eindimensionalen
Potential $V(x)$, gegeben durch
\begin{equation}
  V(x) = \frac{1}{2}kx^2 \, .
\end{equation}
\renewcommand{\labelenumi}{\alph{enumi})}
\renewcommand{\labelenumii}{(\roman{enumii})}
\begin{enumerate}
    \item Leiten Sie die Bewegungsgleichung für die Bahn $x(t)$ des Teilchens mit Hilfe des
    zweiten Newtonschen Axioms her.
    \item Lösen Sie die Bewegungsgleichung mit einem geeigneten Ansatz für die nachfolgenden
    Anfangsbedingungen:
    \begin{enumerate}
      \item $x(t=0) = 0$ $\: \:$und $\dot{x}(t=0) = v_0$
      \item $x(t=0) = x_0$ und $\dot{x}(t=0) = 0$
      \item $x(t=0) = x_0$ und $\dot{x}(t=0) = v_0$
    \end{enumerate}
\end{enumerate}
\end{aufgabe}

% \loesung{
%
% }
%
% \setcounter{punkte}{5}
% Betrachten Sie einen elastischen zentralen Stoß zweier Massen $m_1$ und $m_2$.
% Die Masse $m_1$ wird als bekannt vorausgesetzt und stellt somit die Referenzmasse dar.
% Vor dem Stoß bewegen sich die Massen mit der jeweiligen Geschwindigkeit
% $v_1$ und $v_2$ aufeinander zu. Nach dem Stoß bewegen sich die beiden Massen jeweils mit
% den Geschwindigkeiten $v'_1$ und $v'_2$. Alle Bewegungen finden nur in x-Richtung statt.\\\\
% In einem Experiment messen Sie $m_1$, $v_1$, $v_2$, $v'_1$ und $v'_2$ und
% wollen daraus nun den Wert $m_2$ der zweiten Masse bestimmen.
% Drücken Sie also $m_2$ durch $m_1$, $v_1$, $v_2$, $v'_1$ und $v'_2$ aus.
