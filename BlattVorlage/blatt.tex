%%%%%%%%%%%%%%%%%%%%%%%%%%%%%%%%%%%%%%%%%%%
% Vorlage für Übungsblätter des
% Lehrstuhls für Theoretische Physik I
%%%%%%%%%%%%%%%%%%%%%%%%%%%%%%%%%%%%%%%%%%%

%% ----------------- %%
%%     Dokument      %%
%% ----------------- %%
% !TeX document-id = {f1565f80-4f14-4892-b1e7-460d05a69803}
\documentclass[a4paper,11pt]{article}
\usepackage{ngerman}
\usepackage{bbm,amsmath}
\usepackage[T1]{fontenc}
\usepackage[utf8]{inputenc}
\usepackage{graphicx}
\usepackage{paralist}
\usepackage{textcomp}
\usepackage{epsfig}
\usepackage{amssymb}
\usepackage{comment}
\usepackage{dsfont}
\usepackage{type1cm}
\usepackage{nicefrac}
\usepackage{braket}
\usepackage{physics}
\usepackage{hyperref}
\hypersetup{colorlinks=false,
            pdfcreator=pdf-LaTeX}
\usepackage{pgf}
\usepackage{pgfplots}
\pgfplotsset{compat=newest}
\usepackage{tikz}
\usepackage{siunitx}
\sisetup{number-unit-product = \text{ }}
\sisetup{exponent-product = \cdot, output-product = \cdot}

%%  Übungs-Paket einbinden, Makefile entscheidet darüber, ob mit/ohne Lsgn.
\ifdefined\issolution
  \usepackage[loesung]{uebung}
  \newcommand{\pkt}[1]{\textbf{#1\,P.}}
\else
  \usepackage{uebung}
  \newcommand{\pkt}[1]{}
\fi

%%  Makros
\newcommand {\e}  {\mathrm{e}}
\newcommand {\im} {\mathrm{i}}
\newcommand {\ud} {\mathrm{d}}
\DeclareMathOperator{\dx}{d\!}
\newcommand{\dpnc}[2]{\ensuremath{\frac{\partial\,#1}{\partial\,#2}}}

%% ----------------- %%
%%      Header       %%
%% ----------------- %%
\setcounter{uebung}{1}	% Nummer des Übungsblatts
\setcounter{section}{0}	% Nummer der letzten Aufgabe (*VORIGES* ÜB)
\setcounter{punkte}{5}	% Default für Punktezahl

% Zum Ändern der Punkte für eine einzelne Aufgabe auf n Punkte vorher
% \setcount{punkte}{n} und danach \setcounter{punkte}{5}

\begin{document}
\uebung{
% Ausgabedatum
16.10.2018
}{
% Abgabedatum
24.10.2018, 12 Uhr
} \noindent

%% ----------------- %%
%%     Aufgaben      %%
%% ----------------- %%

% Aufgaben aus dem Contents Ordner importieren
\setcounter{punkte}{5}
\begin{aufgabe}{Das geheime Liebesleben der Buckelwale}

\end{aufgabe}

\loesung{

}

\setcounter{punkte}{5}
\setcounter{punkte}{3}
\begin{aufgabe}{Elastischer Stoß}
Betrachten Sie einen elastischen zentralen Stoß zweier Massen $m_1$ und $m_2$.
Die Masse $m_1$ wird als bekannt vorausgesetzt und stellt somit die Referenzmasse dar.
Vor dem Stoß bewegen sich die Massen mit der jeweiligen Geschwindigkeit
$v_1$ und $v_2$ aufeinander zu. Nach dem Stoß bewegen sich die beiden Massen jeweils mit
den Geschwindigkeiten $v'_1$ und $v'_2$. Alle Bewegungen finden nur in x-Richtung statt.\\\\
In einem Experiment messen Sie $m_1$, $v_1$, $v_2$, $v'_1$ und $v'_2$ und
wollen daraus nun den Wert $m_2$ der zweiten Masse bestimmen.
Drücken Sie also $m_2$ durch $m_1$, $v_1$, $v_2$, $v'_1$ und $v'_2$ aus.
\end{aufgabe}

% \loesung{
%
% }
%
% \setcounter{punkte}{5}

\setcounter{punkte}{5}
\begin{aufgabe}{Trauriger Platzhalter}

\end{aufgabe}

\loesung{

}

\setcounter{punkte}{5}

% \setcounter{punkte}{5}
\begin{aufgabe}{Die monstermäßige Aufgabe, die keiner hinbekommt}

\end{aufgabe}

\loesung{

}

\setcounter{punkte}{5}

\end{document}
